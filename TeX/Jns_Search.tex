%%%%%%%%%%%%%%%%%%%%%%%%%%%%%%%%%%%%%%%%%%%%%%%%%%%%%%%%%%%%%%
% HUMANOIDS VERSION
%%%%%%%%%%%%%%%%%%%%%%%%%%%%%%%%%%%%%%%%%%%%%%%%%%%%%%%%%%%%%%
%\documentclass[letterpaper, 10 pt, conference]{ieeeconf}
\documentclass[conference]{IEEEtran}
\IEEEoverridecommandlockouts 
\usepackage{cite}
\usepackage[cmex10]{amsmath}
\usepackage{amssymb}
%\usepackage{algorithmic}
\usepackage{array}
\usepackage{mdwmath}
\usepackage{mdwtab}
\usepackage{eqparbox}
\usepackage{graphicx}
\usepackage[]{subfigure}
\usepackage{url}
\usepackage[algoruled,vlined,linesnumbered]{algorithm2e}
\usepackage{verbatim}

\hyphenation{op-tical net-works semi-conduc-tor}

% *********************************************************
% Math characters shortcuts
\newcommand{\J}{\ensuremath{J}}
\newcommand{\Jps}{\ensuremath{J^{\dagger}}}
\newcommand{\dx}{\ensuremath{\dot{x}}}
\newcommand{\dq}{\ensuremath{\dot{q}}}
\newcommand{\q}{\ensuremath{q}}
\newcommand{\JsPath}{\ensuremath{\mathcal{Q}}}
\newcommand{\WsPath}{\ensuremath{\mathcal{P}}} % Workspace path
\newcommand{\Wsp}{\ensuremath{p}} % Workspace point
\newcommand{\nsb}{\ensuremath{\hat{e}}} % Nullspace basis
\newcommand{\nsc}{\ensuremath{w}}  % Nullspace coefficients
\newcommand{\nsS}{\ensuremath{\mathcal{NS}}}  % Nullspace Set - vector of queues
\newcommand{\nss}{\ensuremath{ns}}  % Nullspace element single queue
\begin{document}

% *********************************************************
% Paper Info
\title{{D}eterministic {M}otion {P}lanning for {R}edundant {R}obots along {E}nd-{E}ffector {P}aths}
\author{Ana Huam\'an Quispe and Mike Stilman% <-this % stops a space
  \thanks{The authors are with the Center for Robotics and Intelligent
    Machines at the Georgia Institute of Technology, Atlanta, GA
    30332, USA. {\tt\small ahuaman3@gatech.edu}, {\tt\small mstilman@cc.gatech.edu}}}
\maketitle

% *********************************************************
\begin{abstract}
%\boldmath
In this paper we propose a deterministic approach to solve the \emph{Motion Planning
along End-Effector Paths problem} (MPEP) for redundant manipulators. Most of the existing
approaches are based on local optimization, hence they do not
offer global guarantees of finding a path if it exists. Our proposed method is resolution complete.
This feature is achieved by discretizing the Jacobian nullspace at each waypoint and selecting the 
next configuration according to a given heuristic function. To escape from possible local minima, our
algorithm implements a backtracking strategy that allows our planner to recover from erroneous
previous configuration choices by performing a breadth-first backwards search procedure. We present the 
results of simulated experiments performed with diverse manipulators and a humanoid robot.  

\end{abstract}

% **********************************************************
\section{Introduction}
\cite{oriolo-ns-2002} Task redundancy is a desirable feature in robotic manipulators. The 
additional degrees of freedom allow the robot to not only achieve
a primary goal, such as following an end-effector trajectory, but 
it also endows the system with multiple ways to execute the same 
task successfully. Diverse uses of redundancy include obstacle 
avoidance in cluttered environments, joint velocity control,
singularity avoidance, torque minimization among others.

Although redundancy is advantageous, it comes at a
 cost. The complexity of the manipulator increases, which means
that the solution of the \emph{inverse kinematics problem} does not have
a closed-form solution except for particular configurations.
Multiple approaches have been proposed, most of them tailored to
solve the problem under specific assumptions and hence assessing different
success criteria depending on the particular task \cite{hooper-ns-1995}. 

The inverse kinematics problem was initially analyzed under an industrial
perspective, where criteria such as minimization of kinetic energy and manipulatability
were of vital interest. Since robots are slowly making the transition
of factories to more challenging environment such as homes, the manipulator requirements
are more complex, hence redundancy is a useful and needed feature.
 
The present paper focuses on solving a task typically encountered by a 
manipulator: Tracking an end-effector trajectory
such that the manipulator avoids collisions in a static
environment. This kind of task is important not only in the context of the robot
accomplishing a task but it is also a helpful tool for robot learning, facilitating
the demonstration of a task executed by a human (in which only the end-effector movement becomes 
the key feature, leaving to the robot the task of figuring out how to replicate the movement with
its own particular kinematic configuration that may or may not be anthropomorphic). 

We present background inverse kinematic theory and some of the existing approaches in Section \ref{sec:Background}. 
On Section \ref{sec:ProposedAlgorithm} we describe our proposed
algorithm. Results of series of simulated experiments are presented in section \ref{sec:Experiments}. 
We end this paper with conclusions and future work in Section \ref{sec:Conclusions}.
% ...............................................
% Cover Figure
\begin{figure}[]
  \centering
  \subfigure[Workspace Path]{
    \includegraphics[height=100pt]{figures/Cover_Barret_1_WorkspacePath.png} 	
  }
  \subfigure[Execution]{
    \includegraphics[height=100pt]{figures/Cover_Barret_1_400x450.png} 
  } 
  \caption{ Experiment with a 7-DOF BarretWMA Arm}
  \label{fig:CoverFigure}
\end{figure}


% ***********************************************************
\section{Background}
\label{sec:Background}
The forward kinematics of a manipulator is given as follows:

\begin{equation}
x = f(\q)
\label{eq:FK}
\end{equation}

and its differential version is expressed as:

\begin{equation}
\dot{x} = \J \dq
\label{eq:Diff_FK}
\end{equation}

where $x \in R^{m}$, $\q \in R^{n}$, and $m$ and $n$ are the dimensions
of the task and joint space respectively. If $m < n$, the manipulator is
\emph{task redundant}. 

Manipulator tasks are varied; however, here we briefly describe two of the most
common (as it is similary explained in \cite{seereeram-ns-1995}):

\begin{itemize}
\item{\textit{End-Point Goal Task:} The manipulator is given start and end task space points. 
The task is to find a sequence of joint space configurations that connect them. 
This is most known as a \textit{path planning problem} } 
\item{\textit{Point-to-Point Task:} The manipulator is given a task space trajectory to be mapped
to a joint space trajectory. This is most known as a \textit{tracking problem}, which will be 
the focus of this paper.}
\end{itemize}

The inverse kinematics problem consists on finding a mapping such that $\q = f^{-1}(x)$. For
redundant manipulators there are potentially infinite values $\q$ for a given $x$, which can be obtained
by solving (\ref{eq:Diff_FK}). A particular solution is given by:

\begin{equation}
\dq = \Jps \dx
\label{eq:IK_ParticularSol}
\end{equation}

where $\Jps$ is a generalized inverse that can be chosen to minimize a specific criterion. 
In \cite{Whitney-motionRate-1969}, Whitney used the Moore-Penrose pseudo-inverse to minimize the norm of $\dq$. 
Whitney further proposed to use a pseudo-inverse Jacobian weighted with the inertia matrix in 
order to minimize the kinematic energy of the system. A generalization of this is the weighted 
least-norm method (WLM) presented in \cite{chan-ns-1995}, which proved to be particularly effective 
to avoid joint limits. Recent extensions to this approach include the work of Xiang with GWLM 
\cite{xiang-ns-2010}.  
\medskip

(\ref{eq:IK_ParticularSol}) is a particular solution of \ref{eq:Diff_FK}. In general, the set of possible solutions can be expressed as:
\begin{equation}
\dq = \Jps \dx + (I - \Jps \J)\dq_{0}
\label{eq:IK_GeneralSol}
\end{equation}

The second right-hand term represents the homogeneous solutions or \textit{self-motions}, that is, motions 
in the configuration space that do not produce motions in the task space. 

Homogeneous solutions are configurations in the Jacobian nullspace. These can be found by projecting 
an arbitrary vector $\dq_{0}$ on the nullspace, such as in (\ref{eq:IK_GeneralSol}) where the columns 
of $(I - \Jps \J)$ are the basis of the nullspace of $\J$. Most of the proposed methods differ in 
their choosing of $\dq_{0}$. The most widely used method is the \textit{Gradient projection method} 
(GPM) proposed by Liegeois in \cite{liegeois-ns-1977}. The GPM method consists on defining an 
optimization function $H(\q)$ to be minimized. By defining:

\begin{equation}
\dq_{0} = -\alpha \dfrac{\partial H}{\partial \q}
\end{equation}

it produces solutions that move the manipulator away from undesirable configurations. 
Diverse $H$ functions have been used in the literature, such as the measure manipulatability
 in \cite{yoshikawa-ns-1985}, used to avoid singularities. Other applications include obstacle
avoidance \cite{klein-ns-1985}, minimization of torques \cite{hollerbach-ns-1985}, avoidance
of joint limits \cite{liegeois-ns-1977}, among others. An excellent overview of these early
methods can be found in \cite{siciliano-ns-1990}. For problems with more than one subtask,
nested approaches have been proposed \cite{chiaverini-ns-1997}\cite{nakamura-ns-1987}, 
based on the relative priority between each subtask. Other methods instead define a weigthed
sum of optimization functions, such as in \cite{buss-ns-2006}. 

There are other approaches that tackle the redundancy problem by defining additional constraints
 such that the problem is no longer redundant. Examples of these are the Extended Jacobian 
\cite{baillieul-ns-1985} and the Augmented Jacobian \cite{sciavicco-ns-1988}\cite{egeland-ns-1987}.
\medskip 

The methods mentioned above solve different kinds of problems; however all of them share the same 
weakness of being local, which precludes them to fall in local minima. There are a few global 
approaches in the literature, but these are rarely used in practice due to the intensive computation required
(Add citations). In the next section we propose a method that attempts to escape the local minima curse by 
using backtracking in the discretized nullspace of the manipulator.

% *********************************************************
\section{Proposed Algorithm}
\label{sec:ProposedAlgorithm}

% ###################
\subsection{Basics}
We saw in section \ref{sec:Background} that the solution to (\ref{eq:Diff_FK}) has the general form:

\begin{equation}
\dq = \Jps \dx + (I - \Jps \J)\dq_{0}
\label{eq:IK_GeneralSol_2}
\end{equation}

where the second right-hand term represents the self-motions in the Jacobian nullspace, which is a
subspace of dimensions $(n-m)$. Hence, we can write (\ref{eq:IK_GeneralSol_2}) equivalently as:

\begin{equation}
\dq = \Jps \dx + \displaystyle \sum_{i=1}^{n-m} \nsc_{i}\nsb_{i}
\label{eq:Proposed_Eq1}
\end{equation}

where:
\begin{itemize}
\item{ $\nsb_{i}$: Normalized basis of the Jacobian nullspace}
\item{$\nsc_{i}$: Coefficients of each $\nsb_{i}$}
\end{itemize}

Both representations are equivalent, however (\ref{eq:Proposed_Eq1}) is of special interest to us since
instead of having to define a vector $\q_{0} \in R^{n}$, we define the homogeneous solution in terms of 
the $(n-m)$ coefficient $\nsc_{i}$. The problem, however still has potentially infinite solutions.

Our approach proposes, instead of finding a $\q_{0}$ vector (as GPM does), to effectuate a 
search in the discretized Jacobian nullspace. The discretization is carried out by selecting a range of
values for each $\nsc_{i}$. Notice that, since we are considering the tracking of a task space \textit{trajectory}, 
we do only consider homogeneous solutions that are realizable in one time step. To make this clear, please refer 
to Fig.\ref{fig:SelfMotion_Illustration}. Figures \ref{fig:SelfMotion_Illustration}.a and 
\ref{fig:SelfMotion_Illustration}.c show the one-step self-motions that can be achieved in one time step. Figures \ref{fig:SelfMotion_Illustration}.b 
and \ref{fig:SelfMotion_Illustration}.d show the subset of additional self-motions corresponding to these 
that would require more than one time-step to be executed, hence they are not considered in 
the nullspace search explained in this paper\footnote{Notice that in the
general case of tracking a \textit{path}, \textit{all} the self-motion configurations should also be considered. We
are currently working on this topic}.     

% ...............................................
% Self-motion limitation
\begin{figure}[]
  \centering
  \subfigure[One-step self-motion A]{
    \includegraphics[height=90pt]{figures/singleNS_illustration1.png} }
  \subfigure[Set of self-motions A]{
    \includegraphics[height=90pt]{figures/completeNS_illustration1.png}
  } \\
  \subfigure[One-step self-motion B]{
    \includegraphics[height=90pt]{figures/singleNS_illustration2.png} 	
  }
  \subfigure[Set of self-motions B]{
    \includegraphics[height=90pt]{figures/completeNS_illustration2.png}
  }  
  \caption{ Illustration of self-motion considered}
  \label{fig:SelfMotion_Illustration}
\end{figure}

Once the set of self-motions is generated we proceed to eliminate the configurations that are in collision, keeping only
the set of \textit{valid} configurations, from which any can be a possible solution for the task space point evaluated.
Rather than picking a random configuration from the nullspace set, we choose a configuration such that it maximizes a
performance optimization function. Previous work such as \cite{kapoor-ns-1998} suggests a serie of diverse functions to assess
the desirability of each configuration. In particular, we have evaluated the so-called Joint Range Availability measurement (JRA):

\begin{equation}
JRA(\q)=  \displaystyle \sum_{i=1}^{n} \dfrac{ \q_{i} - {\bar{\q}}_{i} }{ {\q_{i}}_{Max} - {\q_{i}}_{Min} }
\end{equation}

in our experiments. We save the nullspace sets as priority queues ordered w.r.t. the JRA measure of each configuration. The approach explained so far is shown in Algorithm \ref{alg:Simple_Track}

% -------------------------------------------------------
% SimpleTrack Algorithm
\begin{algorithm}
\dontprintsemicolon
\KwIn{ Workspace trajectory $\WsPath$ and initial joint configuration $\q_{0}$ } 
\KwOut{ Jointspace trajectory $\JsPath$ } 
\SetKwFunction{GenerateSol}{Generate\_Sol}
\SetKwFunction{FALSE}{false}
\SetKwFunction{TRUE}{true}

$\q \leftarrow \q_{0}$ \;
\BlankLine
\For{ $i \leftarrow 1$ \KwTo $\WsPath$.\FuncSty{size()} }{
	\eIf{ \GenerateSol{$\q$, $\WsPath[i]$, $\nsS[i]$ } {\bf is} \FALSE }{
		\label{algLine:BacktrackLoc} \Return \FALSE \;
	} {
		$\q \leftarrow \nsS[i][0]$ \tcp*{Top priority element} \;
		$\JsPath$.\FuncSty{push\_back($\q$)} \;
	}
}
\BlankLine

\Return \TRUE \;
\caption{ Simple\_Track( $\WsPath$, $\q_{0}$, $\JsPath$ ) }
\label{alg:Simple_Track}
\end{algorithm}
% -------------------------------------------------------


% -------------------------------------------------------
% GenerateSol Algorithm
\begin{algorithm}
\dontprintsemicolon
\KwIn{ Current joint configuration $\q$, next workspace point $\Wsp$ } 
\KwOut{ Nullspace queue $\nss$ corresponding to $\q$ } 
\SetKwFunction{GetPoseError}{GetPoseError}
\SetKwFunction{ToTaskSpace}{ToTaskSpace}
\SetKwFunction{IsInCollision}{InCollision}
\SetKwFunction{IsInLimits}{InLimits}
\SetKwFunction{JRA}{JRA}
\SetKwFunction{FALSE}{false}
\SetKwFunction{TRUE}{true}
\BlankLine

$ds \leftarrow (\Wsp - \ToTaskSpace{\q})$ \;
$\q_{p} \leftarrow \Jps(q)\cdot ds$ \tcp*{Least-norm sol}\;
$ \nsb = \J(q).\FuncSty{kernel()}$ \tcp*{J nullspace basis}\;
\BlankLine
\ForAll{combinations of $w_{j}$}{
 $\q_{ns} \leftarrow \q + \q_{p} + \sum_{j=1}^{m-n}w_{j} \nsb_{j}$ \;
 \BlankLine
 \If{$\parallel \Wsp$ - \ToTaskSpace{$\q_{ns}$}$\parallel < $ \ArgSty{thresh} }{
     \If{ \IsInCollision{$\q_{ns}$} {\bf is} \FALSE \bf{and} \IsInLimits{$\q_{ns}$} {\bf is} \TRUE }{
     $\nss$.\FuncSty{Push\_Queue($\q_{ns}$,\JRA{$\q_{ns}$}) } \;
     } 
	}
}

\eIf{$\nss$.\FuncSty{size()} $> 0$}{
	\Return \TRUE \; 
}(){
	\Return \FALSE \;
}
\caption{ Generate\_Sol( $\q$, $\Wsp$, $\nss$ ) }
\label{alg:GenerateSol}
\end{algorithm}
% -------------------------------------------------------

Algorithm \ref{alg:Simple_Track} is prone to fall in local minima. This
can be easily seen in Line \ref{algLine:BacktrackLoc} in Algorithm \ref{alg:Simple_Track}, which 
reports failure if a solution is not found for a workspace point. We know that a solution might exist 
if previous configurations were chosen differently. Hence, our method improves this by implementing a
\textit{backtracking schema} that operates whenever local solutions fail. 

% ###################
\subsection{Backtracking}
Algorithm \ref{alg:Backtrack} implements a recursive backtracking procedure. It accepts as
input the current configuration, the stored nullspace sets of the trajectory mapped so far
and the maximum number of backtracking steps allowed. When a failure is detected at time $i$, our
algorithm goes back one step ($i-1$), pop up the failing configuration and choose a different one 
from the $\nsS[i-1]$. Notice that since $\nsS$ stores the nullspace sets as priority queues, the next
configuration to be evaluated is the next best, according to our optimization function. In case the 
backtrack fails again in producing a valid solution, it backtracks to $i-2$ and it will repeat
the procedure until it finds a solution or until it reaches the maximum allowed number of backtracking steps.

Of course, backtracking can be computationally intensive if the depth is too big. In order to avoid
backtracking as much as possible, we must use a reasonable heuristic function such that backtrack is only
used in extreme cases. In our experiments we only used the $JRA$ function, which seemed to produce adequate results.  

Our proposed method is shown in Algorithm \ref{alg:Complete_Track}. The recursive backtracking routine (which actually
implements a breadth-first search) is presented in Algorithm \ref{alg:Backtrack}. Notice that the latter algorithm
depends on the \texttt{ForwardSearch} routine(Algorithm \ref{alg:ForwardSearch}) which searches for a solution from
the backtracked configuration $(\ArgSty{i}-\ArgSty{step})$. Also note that each time a backtrack is executed, the nullspace sets generated previously
from $(\ArgSty{i}-\ArgSty{step}-1)$ to $i$ are cleared and updated according to the new configuration found in  $(\ArgSty{i}- \ArgSty{step})$.

% -------------------------------------------------------
% CompleteTrack Algorithm
\begin{algorithm}[h]
\dontprintsemicolon
\KwIn{ Workspace trajectory $\WsPath$ and initial joint configuration $\q_{0}$,
maximum backtrack allowed \ArgSty{maxStep} } 
\KwOut{ Jointspace trajectory $\JsPath$ } 
\SetKwFunction{GenerateSol}{Generate\_Sol}
\SetKwFunction{Backtrack}{Backtrack}
\SetKwFunction{Update}{Update}
\SetKwFunction{FALSE}{false}
\SetKwFunction{TRUE}{true}

$\q \leftarrow \q_{0}$ \;
\BlankLine
\For{ $i \leftarrow 1$ \KwTo $\WsPath$.\FuncSty{size()} }{
	\eIf{ \GenerateSol{$\q$, $\WsPath[i]$, $\nsS[i]$ } {\bf is} \FALSE }{
		\ArgSty{step} $\leftarrow 1$ \;
		b = \Backtrack{i, step, maxStep, $\nsS$, $\WsPath$} \;
		\eIf{ b is \FALSE}{
			\Return \FALSE \;
		} {
			\Update{$\JsPath$, \ArgSty{i}, \ArgSty{step}, $\nsS$} \;
		}
	} {
		$\q \leftarrow \nsS[i][0]$ \;
		$\JsPath$.\FuncSty{push\_back($\q$)} \;
	}
}
\BlankLine
\Return \TRUE \;
\caption{ Complete\_Track( $\WsPath$, $\q_{0}$, $\JsPath$, \ArgSty{maxStep} ) }
\label{alg:Complete_Track}
\end{algorithm}
% -------------------------------------------------------


% -------------------------------------------------------
% Backtrack Algorithm
\begin{algorithm}
\dontprintsemicolon
\KwIn{ current index $i$, current backtrack \ArgSty{step}, maximum backtrack \ArgSty{maxStep},
nullspace sets $\nsS$, workspace trajectory $\WsPath$ } 
\KwOut{ Updated $\nsS$ } 
\SetKwFunction{Backtrack}{Backtrack}
\SetKwFunction{ForwardSearch}{ForwardSearch}
\SetKwFunction{FALSE}{false}
\SetKwFunction{TRUE}{true}

\eIf{ step $<$ maxStep}{
		\eIf{ \ForwardSearch{i, step, maxStep, $\nsS$, $\WsPath$} {\bf is} \FALSE }{
			\Return \Backtrack{i, step++, maxStep, $\nsS$, $\WsPath$} \;
		}{
		\Return \TRUE \;	
		}	
}{
	\Return \ForwardSearch{i, step, maxStep, $\nsS$, $\WsPath$} \;
}
\Return \TRUE \;
\caption{ Backtrack( \ArgSty{i}, \ArgSty{step}, \ArgSty{maxStep}, $\nsS$, $\WsPath$ ) }
\label{alg:Backtrack}
\end{algorithm}
% -------------------------------------------------------

% -------------------------------------------------------
% ForwardSearch Algorithm
\begin{algorithm}
\dontprintsemicolon
\KwIn{ current index $i$, current backtrack \ArgSty{step}, maximum backtrack \ArgSty{maxStep},
nullspace sets $\nsS$ } 
\KwOut{ Updated $\nsS$ } 
\SetKwFunction{ForwardSearch}{ForwardSearch}
\SetKwFunction{ForwardClearNS}{ForwardClear\_NS}
\SetKwFunction{GenerateSol}{Generate\_Sol}
\SetKwFunction{FALSE}{false}
\SetKwFunction{TRUE}{true}

\ForwardClearNS{i, step, $\nsS$} \;
\ArgSty{j} $\leftarrow$ \ArgSty{i - step} \;
\ArgSty{found} $\leftarrow$ \FALSE \;
\While{ $\nsS[j]$.\FuncSty{size()}$> 0$ {\bf and} \ArgSty{found} {\bf is} \FALSE }{
	b $\leftarrow$ \GenerateSol{$\nsS[j][0]$, $\WsPath[j+1]$, $\nsS[j+1]$}\;
	\eIf{ b is \TRUE}{
		\eIf{ step $> 1$}{
			\ArgSty{found} $\leftarrow$ \ForwardSearch{i, \ArgSty{step-1}, \ArgSty{maxStep}, $\nsS$, $\WsPath$} \;
		} {
			\ArgSty{found} $\leftarrow$ \TRUE \;
		}
	}{
		$\nsS[j]$.\FuncSty{Pop\_Queue()} \;
	}
}
\BlankLine
\eIf{ \ArgSty{found} {\bf is} \FALSE}{
	\eIf{ step {\bf is} maxStep }{
		\Return \FALSE \;
	} {
		$\nsS[j+1]$.\FuncSty{Pop\_Queue()} \;
	}
} {
	\Return \TRUE \;
}
\caption{ ForwardSearch( \ArgSty{i}, \ArgSty{step}, \ArgSty{maxStep}, $\nsS$, $\WsPath$) }
\label{alg:ForwardSearch}
\end{algorithm}
% -------------------------------------------------------

% -------------------------------------------------------
\begin{procedure}
\dontprintsemicolon
\SetNoline
%\KwIn{ current index $i$, backtrack performed \ArgSty{step},
%nullspace sets $\nsS$ } 
%\KwOut{ Updated $\JsPath$ after backtracking } 
\SetKwFunction{FALSE}{false}
\SetKwFunction{TRUE}{true}

\For{ $j \leftarrow 1$ \KwTo \ArgSty{step} }{
	\emph{// Update the changed joint trajectory points} \;
	$\JsPath[i-j] \leftarrow \nsS[i-j][0]$\;
}
\caption{ Update($\JsPath$, \ArgSty{i}, \ArgSty{step}, $\nsS$) } 
\label{proc:Update}
\end{procedure}
% -------------------------------------------------------

% -------------------------------------------------------
\begin{procedure}
\dontprintsemicolon
\SetNoline
\SetKwFunction{FALSE}{false}
\SetKwFunction{TRUE}{true}
\If{ \ArgSty{step} $> 1$ }{
	\For{$j \leftarrow 1$ \KwTo \ArgSty{step}}{
	$\nsS[i-j]$.\FuncSty{clear()}\;		
	}
}
\caption{ForwardClearNS(\ArgSty{i}, \ArgSty{step}, $\nsS$)} 
\label{proc:ForwardClearNS}
\end{procedure}
% -------------------------------------------------------

% *********************************************************
\section{Experiments and Results}
\label{sec:Experiments}
In this section we present a series of simulated experiments
with diverse manipulators. The simulations were made using the
integrated GRIP + DART\footnote{ GRIP (\url{https://github.com/golems/grip})
is a robotics simulator based on the physical engine DART (\url{https://github.com/golems/dart}).
Both packages are OpenSource projects currently developed at Georgia Tech} 
software platform. For collision detection we used the VCOLLIDE package\cite{Hudson97v-collide:accelerated}. All the experiments were
executed on a Intel 2 Core Duo (2.80GHz).

% ###############################
\subsection{Preliminaries}
Before presenting the experiments, some details should be mentioned:

\begin{itemize}
\item{All the experiments presented are \textit{end-effector position
tracking} problems. We chose to focus this paper on
position (rather than pose) tracking since the manipulator
has more redundant degrees of freedom with respect to
this kind of tasks. Having less constraints poses a more
challenging scenario to test our algorithm.}
\item{The workspace trajectories used in the experiments were generated
with a low-level workspace planner described in \cite{achq-wsp-2012}}
\item{All the kinematics evaluations were handled by the DART dynamics package. Additional linear algebra calculations (such as the nullspace basis or the Jacobian pseudo-inverse) were performed using the Eigen\footnote{\url{http://eigen.tuxfamily.org}} C++ library.}
\item{All the experiments used a discretization of $10$ values per each $\nsc_{i}$. The range of each $\nsc_{i}$ was $[-10,10]$. The nullspace basis $\nsb_{i}$ were normalized to have the same relative norm w.r.t each other.}
\end{itemize}

% ###############################
\subsection{Experiment 1: LWA3}
Our first series of experiments were done with a 7 DOF
LWA3 arm, as shown in Fig.\ref{fig:Experiment1_LWA3}. The initial configuration of the
robot has its end effector below the cabinet. The workspace
trajectory to follow is shown in Fig.\ref{fig:Experiment1_LWA3}(b). Notice that the
end-effector trajectory is very close to the cabinet, hence the
tracking is harder since there are more chances of the manipulator
links colliding with the object. The resulting execution is
shown in Fig.\ref{fig:Experiment1_LWA3}(c).

% ...............................................
% Experiment 1: LWA3
\begin{figure}[]
  \centering
  \subfigure[Workspace trajectory]{
    \includegraphics[width=0.45\columnwidth, height=80pt]{figures/LWA3_Experiment/Experiment_LWA3_Path_2_WS.png} 
  }
  \subfigure[Joint trajectory]{
    \includegraphics[width=0.45\columnwidth]{figures/LWA3_Experiment/Experiment_LWA3_Path_2_Joints.png} 	
  }
  \subfigure[Execution]{
    \includegraphics[height=100pt]{figures/LWA3_Experiment/Experiment_LWA3_Path_2_Layered_720x540.png} 	
  }
  \caption{ Experiment with a 7-DOF LWA3 Schunk arm}
  \label{fig:Experiment1_LWA3}
\end{figure}


% ###############################
\subsection{Experiment 2: BarretWMA}
Our second experiment was done with a 7 DOF BarretWMA
manipulator. The scenary kept the same as in the experiment
with the LWA3 but the initial configuration of the arm changed
as it is shown in Fig.4. The arm has its end effector located
on top of the cabinet and its target location is resting on top
of the table. The resulting trajectory is shown in Fig.\ref{fig:Experiment2_Barret}.(c).
In this execution, two backtrack calls were needed (at time
step 3 and 31 out of the total trajectory of 35 points) both of
them requiring a depth of 2. Notice that a simple search such
as GPM is not able to solve this problem for this particular
workspace to follow.

% ...............................................
% Experiment 2: Barret Arm

\begin{figure}[]
  \centering
  \subfigure[Workspace path $(x,y,z)$]{
    \includegraphics[height=100pt]{figures/Barret_Experiment/Experiment_Barret_Path_2_WS.png} 
  } 
  \subfigure[Tracking]{
    \includegraphics[width=0.45\columnwidth]{figures/Barret_Experiment/Experiment_Barret_Path_2_Joints.png} 	
  }
  \subfigure[Execution]{
    \includegraphics[width=0.45\columnwidth]{figures/Barret_Experiment/Experiment_Barret_Path_2_Layered_560x560.png} 	
  }
  \caption{ Experiment with a 7-DOF BarretWMA}
  \label{fig:Experiment2_Barret}
\end{figure}


% ###############################
\subsection{Experiment 3: Mitsubishi}

\subsubsection{Case 1: Straight Line}
% ...............................................
% Experiment 3A: Mitsubishi Arm 

\begin{figure}[]
  \centering
  \subfigure[Workspace trajectory]{
    \includegraphics[width=0.45\columnwidth]{figures/Mitsubishi_A_Experiment/wsPath.png} 	
  }
  \subfigure[Tracking]{
    \includegraphics[width=0.45\columnwidth]{figures/Mitsubishi_A_Experiment/jointPlot.png} 	
  }
  \subfigure[Execution]{
    \includegraphics[height=100pt]{figures/Mitsubishi_A_Experiment/movementOverlay.png} 
  }

  \caption{ Experiment with a 6-DOF manipulator}
  \label{fig:Experiment3A_Mitsubishi}
\end{figure}


\subsubsection{Case 2: Straight Line}
% ...............................................
% Experiment 3B: Mitsubishi Arm 

\begin{figure}[]
  \centering
  \subfigure[Workspace trajectory]{
    \includegraphics[width=0.45\columnwidth]{figures/Mitsubishi_B_Experiment/wsPath.png} 	
  }
  \subfigure[Tracking]{
    \includegraphics[width=0.45\columnwidth]{figures/Mitsubishi_B_Experiment/jointPlot.png} 	
  }
  \subfigure[Execution]{
    \includegraphics[height=90pt]{figures/Mitsubishi_B_Experiment/movementOverlay.png} 
  }

  \caption{ Experiment with a 6-DOF manipulator}
  \label{fig:Experiment3B_Mitsubishi}
\end{figure}


% ###############################
\subsection{Experiment 4:  Dual-Armed robot}

% ...............................................
% Experiment 4: LWA4
\begin{figure}[]
  \centering
  \subfigure[Problem setup]{
    \includegraphics[height=90pt]{figures/LWA4_Experiment/config_Problem.png} 
  }
  \subfigure[Workspace traj]{
    \includegraphics[height=90pt]{figures/LWA4_Experiment/BothWSPaths.png} 	
  }\\
  \subfigure[Reach execution]{
    \includegraphics[height=90pt]{figures/LWA4_Experiment/LWA4_1_Layered_small.png} 	
  }
  \subfigure[Move execution]{
    \includegraphics[height=90pt]{figures/LWA4_Experiment/LWA4_2_Layered_small.png} 	
  }
  \caption{ Experiment with a Dual-Armed robot}
  \label{fig:Experiment4_LWA4}
\end{figure}

% *********************************************************
\section{Conclusions and Future Work}
\label{sec:Conclusions}
We have presented an approach to solve the trajectory tracking
problem for redundant manipulators based on the discretization of the Jacobian nullspace
and a backtracking strategy to prevent local minima traps. In contrast
with previous approaches, we exploit the nullspace by considering more than one possible
configuration and allow our method to be more flexible. In contrast with previous
work in which a set of joints were arbitrarily selected as non-redundant, our approach 
uses the nullspace basis of the Jacobian to perform the search with the minimum required
values.

As future work, we intend to devise more simulated experiments with more challenging
scenarios and further test the presented method in a physical manipulator. Additionally,
we plan to test  our approach with \emph{pose tracking} problems to verify its effectiveness.

% *********************************************************
\begin{comment}
\section*{Acknowledgments}
The authors would like to thank all members of the Humanoid Lab
at Georgia Tech for their insightful feedback and continuous support.
\end{comment}
% *********************************************************
%\IEEEtriggeratref{8}
%\IEEEtriggercmd{\enlargethispage{-5in}}
\bibliographystyle{IEEEtran}
\bibliography{JnsReferences}


\end{document}


