
%%%%%%%%%%%%%%%%%%%%%%%%%%%%%%%%%%%%%%%%%%%%%%%%%%%%%%%%%%%%%%
\documentclass[conference]{IEEEtran}
% \documentclass[conference]{../sty/IEEEtran}
\IEEEoverridecommandlockouts 
\usepackage{cite}
\usepackage[cmex10]{amsmath}
\usepackage{algorithmic}
\usepackage{array}
\usepackage{mdwmath}
\usepackage{mdwtab}
\usepackage{eqparbox}
\usepackage[tight,footnotesize]{subfigure}
\usepackage[caption=false,font=footnotesize]{subfig}
\usepackage{fixltx2e}
\usepackage{stfloats}
\usepackage{url}

\hyphenation{op-tical net-works semi-conduc-tor}


\begin{document}

% *********************************************************
% Paper Info
\title{Redundancy Resolution using a Semi-Local Nullspace-based approach}
\author{Ana Huam\'an Quispe and Mike Stilman% <-this % stops a space
  \thanks{The authors are with the Center for Robotics and Intelligent
    Machines at the Georgia Institute of Technology, Atlanta, GA
    30332, USA. {\tt\small ahuaman3@gatech.edu}, {\tt\small mstilman@cc.gatech.edu}}}
\maketitle

% *********************************************************
\begin{abstract}
%\boldmath
This paper revisits the most common methods of tracking a path and 
proposes a semi-local discrete method to find the best available
configuration path that track the input workspace path.
\end{abstract}

% **********************************************************
\section{Introduction}
Redundancy is a desirable feature in robotic manipulators. The 
additional degrees of freedom allow the robot to not only achieve
its primary goal, such as executing a given workspace path, but 
it also endows the system with multiple ways to execute the same 
task successfully. Diverse uses of redundancy include obstacle 
avoidance in cluttered environments, control of joint velocities,
etc.

Although redundancy is advantageous, it comes at a
 cost. The complexity of the manipulator increases, which means
that the solution of the inverse kinematics problem does not have
a closed-form solution except for simple or known configurations.
Multiple approaches have been proposed, most of them tailored to
solve the problem under specific assumptions and assessing different
success criteria \cite{hooper-ns-1995}. 

The present paper focuses on solving the most commonly encountered
situation in daily life: To map a desired workspace path to a path in
configuration space such that it avoids collisions in a static
environment while reducing the energy spent during the execution. We
present a brief overview of the current work on the area in section 
\ref{sec:RelatedWork}. Section \ref{sec:Overview} gives a general overview of
the inverse kinematics problem and based on it we will explain our
approach on section \ref{sec:ProposedAlgorithm}. We present the 
results of a serie of simulated experiments in section \ref{sec:Experiments}
and end the paper with conclusions and future work \ref{sec:Conclusions}.



 

% ***********************************************************
\section{Related Work}
\label{sec:RelatedWork}
The idea of using the Jacobian nullspace to solve the redundancy 
resolution problem was first presented by Whitney \cite{Whitney-motionRate-1969}.
Other authors, such as \cite{liegeois-ns-1977} proposed to use
secondary goal functions to achieve.

% *********************************************************
\section{Overview}
\label{sec:Overview}

% *********************************************************
\section{Proposed Algorithm}
\label{sec:ProposedAlgorithm}


% *********************************************************
\section{Experiments and Results}
\label{sec:Experiments}


% *********************************************************
\section{Conclusions and Future Work}
\label{sec:Conclusions}


% *********************************************************
\section*{Acknowledgments}
The authors would like to thank all members of the Humanoid Lab
at Georgia Tech for their insightful feedback.

% *********************************************************
%\IEEEtriggeratref{8}
%\IEEEtriggercmd{\enlargethispage{-5in}}
\bibliographystyle{IEEEtran}
\bibliography{JnsReferences}


\end{document}


