
%%%%%%%%%%%%%%%%%%%%%%%%%%%%%%%%%%%%%%%%%%%%%%%%%%%%%%%%%%%%%%
\documentclass[conference]{IEEEtran}
% \documentclass[conference]{../sty/IEEEtran}
\IEEEoverridecommandlockouts 
\usepackage{cite}
\usepackage[cmex10]{amsmath}
\usepackage{algorithmic}
\usepackage{array}
\usepackage{mdwmath}
\usepackage{mdwtab}
\usepackage{eqparbox}
\usepackage[tight,footnotesize]{subfigure}
\usepackage[caption=false,font=footnotesize]{subfig}
\usepackage{fixltx2e}
\usepackage{stfloats}
\usepackage{url}

\hyphenation{op-tical net-works semi-conduc-tor}

% *********************************************************
% Math characters shortcuts
\newcommand{\Jps}{ \ensuremath{J^{\dagger}} }
\newcommand{\dx}{ \ensuremath{\dot{x}} }
\newcommand{\dt}{ \ensuremath{\dot{\theta}} }

 
\begin{document}

% *********************************************************
% Paper Info
\title{Redundancy Resolution using a Semi-Local Nullspace-based approach}
\author{Ana Huam\'an Quispe and Mike Stilman% <-this % stops a space
  \thanks{The authors are with the Center for Robotics and Intelligent
    Machines at the Georgia Institute of Technology, Atlanta, GA
    30332, USA. {\tt\small ahuaman3@gatech.edu}, {\tt\small mstilman@cc.gatech.edu}}}
\maketitle

% *********************************************************
\begin{abstract}
%\boldmath
Redundant manipulators pose a non-trivial inverse kinematics problem.
Due to computability restrictions, most of the existing approaches offer
local optimality guarantees, which is satisfactory for a good range of common 
manipulating tasks. In this paper, we propose yet another local method based
on the Jacobian nullspace projection; however our approach offers a 
different treatment of the nullspace, treating it as a search space rather
than a mathematical device. We also suggest an intuitive procedure to select
the joints configuration for the manipulator, based on annealing methods.
\end{abstract}

% **********************************************************
\section{Introduction}
Redundancy is a desirable feature in robotic manipulators. The 
additional degrees of freedom allow the robot to not only achieve
its primary goal, such as executing a given workspace path, but 
it also endows the system with multiple ways to execute the same 
task successfully. Diverse uses of redundancy include obstacle 
avoidance in cluttered environments, control of joint velocities,
avoidance of singularities, etc.

Although redundancy is advantageous, it comes at a
 cost. The complexity of the manipulator increases, which means
that the solution of the inverse kinematics problem does not have
a closed-form solution except for simple or known configurations.
Multiple approaches have been proposed, most of them tailored to
solve the problem under specific assumptions and assessing different
success criteria \cite{hooper-ns-1995}. 

The present paper focuses on solving the most commonly encountered
situation in daily life: To map a desired workspace path to a path in
configuration space such that it avoids collisions in a static
environment while reducing the energy spent during the execution. We
present a brief overview of the current work on the area in section 
\ref{sec:RelatedWork}. Section \ref{sec:Overview} gives a general overview of
the inverse kinematics problem and based on it we will explain our
approach on section \ref{sec:ProposedAlgorithm}. We present the 
results of a serie of simulated experiments in section \ref{sec:Experiments}
and end the paper with conclusions and future work \ref{sec:Conclusions}.


% ***********************************************************
\section{Related Work}
\label{sec:RelatedWork}
The redundance resolution problem has been vastly studied, being the
pioneering work of Whitney in \emph{Motion Rate Control} (\cite{Whitney-motionRate-1969}) perhaps one of the most influentials in all the methods proposed later. 
Whitney proposed the use of the \emph{Jacobian pseudoinverse} to obtain
solutions that minimized the energy spent.  
Other authors, such as \cite{liegeois-ns-1977} proposed to use
secondary goal functions to achieve.

An excellent overview of these methods can be found in \cite{siciliano-ns-1990}

% *********************************************************
\section{Overview}
\label{sec:Overview}

% ---------------------------------------------------------
\subsection{The Inverse Kinematic Problem}
Given a manipulator with $n$ degrees of freedom and a task that
can be represented by $x$. The direct mapping:

\begin{equation}
x = f(\theta) 
\label{eq:DK}
\end{equation}

\begin{equation}
\dt = \Jps \dx + (I - \Jps J)^{-1}\dot{q}_{0}
\label{eq:IK_MinNorm_Solution}
\end{equation}



% *********************************************************
\section{Proposed Algorithm}
\label{sec:ProposedAlgorithm}


% *********************************************************
\section{Experiments and Results}
\label{sec:Experiments}


% *********************************************************
\section{Conclusions and Future Work}
\label{sec:Conclusions}


% *********************************************************
\section*{Acknowledgments}
The authors would like to thank all members of the Humanoid Lab
at Georgia Tech for their insightful feedback.

% *********************************************************
%\IEEEtriggeratref{8}
%\IEEEtriggercmd{\enlargethispage{-5in}}
\bibliographystyle{IEEEtran}
\bibliography{JnsReferences}


\end{document}


